@article{Jolly_2020,
	doi = {10.1088/2040-8986/abad08},
	url = {https://doi.org/10.1088/2040-8986/abad08},
	year = 2020,
	month = {sep},
	publisher = {{IOP} Publishing},
	volume = {22},
	number = {10},
	pages = {103501},
	author = {Spencer W Jolly and Olivier Gobert and Fabien Qu{\'{e}}r{\'{e}}},
	title = {Spatio-temporal characterization of ultrashort laser beams: a tutorial},
	journal = {Journal of Optics},
	abstract = {The temporal characterization of ultrafast laser pulses has become a cornerstone capability of ultrafast optics laboratories and is routine both for optimizing laser pulse duration and designing custom fields. Beyond pure temporal characterization, spatio-temporal characterization provides a more complete measurement of the spatially-varying temporal properties of a laser pulse. These so-called spatio-temporal couplings (STCs) are generally nonseparable chromatic aberrations that can be induced by very common optical elements—for example, diffraction gratings and thick lenses or prisms made from dispersive material. In this tutorial we introduce STCs and a detailed understanding of their behavior in order to have a background knowledge, but also to inform the design of characterization devices. We then overview a broad range of spatio-temporal characterization techniques with a view to mention most techniques, but also to provide greater details on a few chosen methods. The goal is to provide a reference and a comparison of various techniques for newcomers to the field. Lastly, we discuss nuances of analysis and visualization of spatio-temporal data, which is an often underappreciated and non-trivial part of ultrafast pulse characterization.}
}