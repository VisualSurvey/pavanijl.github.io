TY  - JOUR
AU  - Moustakas, Aristides
PY  - 2017
DA  - 2017/05/01
TI  - Spatio-temporal data mining in ecological and veterinary epidemiology
JO  - Stochastic Environmental Research and Risk Assessment
SP  - 829
EP  - 834
VL  - 31
IS  - 4
AB  - Understanding the spread of any disease is a highly complex and interdisciplinary exercise as biological, social, geographic, economic, and medical factors may shape the way a disease moves through a population and options for its eventual control or eradication. Disease spread poses a serious threat in animal and plant health and has implications for ecosystem functioning and species extinctions as well as implications in society through food security and potential disease spread in humans. Space–time epidemiology is based on the concept that various characteristics of the pathogenic agents and the environment interact in order to alter the probability of disease occurrence and form temporal or spatial patterns. Epidemiology aims to identify these patterns and factors, to assess the relevant uncertainty sources, and to describe disease in the population. Thus disease spread at the population level differs from the approach traditionally taken by veterinary practitioners that are principally concerned with the health status of the individual. Patterns of disease occurrence provide insights into which factors may be affecting the health of the population, through investigating which individuals are affected, where are these individuals located and when did they become infected. With the rapid development of smart sensors, social networks, as well as digital maps and remotely-sensed imagery spatio-temporal data are more ubiquitous and richer than ever before. The availability of such large datasets (big data) poses great challenges in data analysis. In addition, increased availability of computing power facilitates the use of computationally-intensive methods for the analysis of such data. Thus new methods as well as case studies are needed to understand veterinary and ecological epidemiology. A special issue aimed to address this topic.
SN  - 1436-3259
UR  - https://doi.org/10.1007/s00477-016-1374-8
DO  - 10.1007/s00477-016-1374-8
ID  - Moustakas2017
ER  - 
