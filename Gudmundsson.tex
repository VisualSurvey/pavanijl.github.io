@article{10.1145/3054132,
author = {Gudmundsson, Joachim and Horton, Michael},
title = {Spatio-Temporal Analysis of Team Sports},
year = {2017},
issue_date = {June 2017},
publisher = {Association for Computing Machinery},
address = {New York, NY, USA},
volume = {50},
number = {2},
issn = {0360-0300},
url = {https://doi.org/10.1145/3054132},
doi = {10.1145/3054132},
abstract = {Team-based invasion sports such as football, basketball, and hockey are similar in the sense that the players are able to move freely around the playing area and that player and team performance cannot be fully analysed without considering the movements and interactions of all players as a group. State-of-the-art object tracking systems now produce spatio-temporal traces of player trajectories with high definition and high frequency, and this, in turn, has facilitated a variety of research efforts, across many disciplines, to extract insight from the trajectories. We survey recent research efforts that use spatio-temporal data from team sports as input and involve non-trivial computation. This article categorises the research efforts in a coherent framework and identifies a number of open research questions.},
journal = {ACM Comput. Surv.},
month = apr,
articleno = {22},
numpages = {34},
keywords = {spatio-temporal data, soccer, spatial subdivision, handball, data mining, hockey, machine learning, Trajectory, basketball, sports analysis, network analysis, american football, football, performance metrics}
}