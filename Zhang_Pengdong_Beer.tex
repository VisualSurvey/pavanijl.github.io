
@Article{ijgi7010031,
AUTHOR = {Zhang, Pengdong and Beernaerts, Jasper and Van de Weghe, Nico},
TITLE = {A Hybrid Approach Combining the Multi-Temporal Scale Spatio-Temporal Network with the Continuous Triangular Model for Exploring Dynamic Interactions in Movement Data: A Case Study of Football},
JOURNAL = {ISPRS International Journal of Geo-Information},
VOLUME = {7},
YEAR = {2018},
NUMBER = {1},
ARTICLE-NUMBER = {31},
URL = {https://www.mdpi.com/2220-9964/7/1/31},
ISSN = {2220-9964},
ABSTRACT = {Benefiting from recent advantages in location-aware technologies, movement data are becoming ubiquitous. Hence, numerous research topics with respect to movement data have been undertaken. Yet, the research of dynamic interactions in movement data is still in its infancy. In this paper, we propose a hybrid approach combining the multi-temporal scale spatio-temporal network (MTSSTN) and the continuous triangular model (CTM) for exploring dynamic interactions in movement data. The approach mainly includes four steps: first, the relative trajectory calculus (RTC) is used to derive three types of interaction patterns; second, for each interaction pattern, a corresponding MTSSTN is generated; third, for each MTSSTN, the interaction intensity measures and three centrality measures (i.e., degree, betweenness and closeness) are calculated; finally, the results are visualized at multiple temporal scales using the CTM and analyzed based on the generated CTM diagrams. Based on the proposed approach, three distinctive aims can be achieved for each interaction pattern at multiple temporal scales: (1) exploring the interaction intensities between any two individuals; (2) exploring the interaction intensities among multiple individuals, and (3) exploring the importance of each individual and identifying the most important individuals. The movement data obtained from a real football match are used as a case study to validate the effectiveness of the proposed approach. The results demonstrate that the proposed approach is useful in exploring dynamic interactions in football movement data and discovering insightful information.},
DOI = {10.3390/ijgi7010031}
}



