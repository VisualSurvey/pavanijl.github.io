
@Article{rs12040608,
AUTHOR = {Xu, Min and Cao, Chunxiang and Jia, Peng},
TITLE = {Mapping Fine-Scale Urban Spatial Population Distribution Based on High-Resolution Stereo Pair Images, Points of Interest, and Land Cover Data},
JOURNAL = {Remote Sensing},
VOLUME = {12},
YEAR = {2020},
NUMBER = {4},
ARTICLE-NUMBER = {608},
URL = {https://www.mdpi.com/2072-4292/12/4/608},
ISSN = {2072-4292},
ABSTRACT = {Fine-scale population distribution is increasingly becoming a research hotspot owing to its high demand in many applied fields. It is of great significance in urban emergency response, disaster assessment, resource allocation, urban planning, market research, and transportation route design. This study employed land cover, building address, and housing price data, and high-resolution stereo pair remote sensing images to simulate fine-scale urban population distribution. We firstly extracted the residential zones on the basis of land cover and Google Earth data, combined them with building information including address and price. Then, we employed the stereo pair analysis method to obtain the building height on the basis of ZY3-02 high-resolution satellite data and transform the building height into building floors. After that, we built a sophisticated, high spatial resolution model of population density. Finally, we evaluated the accuracy of the model using the survey data from 12 communities in the study area. Results demonstrated that the proposed model for spatial fine-scale urban population products yielded more accurate small-area population estimation relative to high-resolution gridded population surface (HGPS). The approach proposed in this study holds potential to improve the precision and automation of high-resolution population estimation.},
DOI = {10.3390/rs12040608}
}



