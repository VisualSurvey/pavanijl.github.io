@article{Li:18,
author = {Zhaoyang Li and Noriaki Miyanaga},
journal = {Opt. Express},
keywords = {Lasers; Ultrafast lasers; Wave propagation ; Femtosecond lasers; Femtosecond pulses; Fourier transform spectroscopy; Laser beams; Optical aberrations; Ultrafast lasers},
number = {7},
pages = {8453--8469},
publisher = {OSA},
title = {Simulating ultra-intense femtosecond lasers in the 3-dimensional space-time domain},
volume = {26},
month = {Apr},
year = {2018},
url = {http://www.opticsexpress.org/abstract.cfm?URI=oe-26-7-8453},
doi = {10.1364/OE.26.008453},
abstract = {Femtosecond petawatt (fs-PW) lasers, with femtosecond pulses and sub-meter-sized beams, could be easily distorted by spatiotemporal coupling (STC). In 2016, a femtosecond terawatt pulsed beam was experimentally reconstructed in the 3-dimensional (3D) space-time domain for the first time, and showing STC induced distortions. Referring to recently developed laser techniques, traditional first-order STCs can be controlled and then removed. However, the complex STC induced by wavefront errors in a meter-sized grating compressor, where the spatial and spectral coordinates of beams and pulses are coupled, would introduce a non-negligible and complicated distortion. Herein, we theoretically simulated this complex STC in the 3D space-time/spectrum domain and presented its evolution with various factors, which opens a new perspective to analyze CPA lasers in the 3D domain.},
}

